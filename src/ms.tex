\RequirePackage[l2tabu, orthodox]{nag}
% Define document class
\documentclass[]{report}
\usepackage{style}

% Begin!
\begin{document}

% Title
\title{\vspace{-15pt}\begin{center}\includegraphics[width=0.4\linewidth]{static/misc/logo.jpg}\end{center}
    \vspace{20pt}
    \begin{center}
        School of Physics \& Astronomy
    \end{center}
    \vspace{20pt}
    \begin{center}
        \LARGE\textbf{Probabilistic modeling of astrophysical time series:
            gravitational microlensing and occultation mapping of planets and moons}
    \end{center}
    \vspace{20pt}
    \begin{center}
        Supervisor: Dr. Martin Dominik
    \end{center}
}

\maketitle
\tableofcontents

% Abstract 
\begin{abstract}
    Scientific progress in modern astronomy research commonly relies on gathering large quantities of
    data using exceedingly precise instruments.
    The process which ``generates'' these data consists of the physical phenomenon of
    interest -- for instance, an exoplanet blocking or twisting the light of a distant star, and
    the noise introduced by the measurement process and the presence of an atmosphere
    The task for an astronomer is to first construct a \emph{model} which describes the entire process
    which generated the data and to then ``fit'' that model to data.
    All models only approximate reality and the researcher has to make a series of decisions during the
    model building process, everything from how to process the raw data to which results to put in
    the abstract of a paper.
    Advancements in computational statistics and machine learning in the past decade or
    so have made it possible to fit ever more complex models to data.
    These models are generally expressed in computer code which may contain complex numerical algorithms,
    such as iterative solvers and numerical integrals.
    In this thesis, I mostly focused on developing methods which enable \emph{statistical inference}
    with these kinds of complex models in two particular domains within astronomy, gravitational
    microlensing and occultation mapping.
    The common theme between these two topics is that both deal with accurately measuring
    the brightness of distant stars as a function of time with the goal of inferring properties of
    exoplanets and stars.
    Broadly speaking, I believe the biggest contribution of this thesis is providing a new lens
    for looking at a particular set of old problems, a lens which incorporates recent advancements
    from statistics, machine learning and computer science.
    More specifically, I have developed a an open-source software
    package \textsf{caustics}\footnote{\url{https://github.com/fbartolic/caustics}} which enables
    fast and accurate computation of binary lens and triple lens microlensing light curves and
    simultaneously provides exact \emph{gradients} of the code outputs with respect to all its inputs.
    This is significant because it for the first time enables the use of modern gradient based
    statistical inference algorithms such as Hamiltonian Monte Carlo with microlensing light curves.
    Microlensing is one of the major goals for the upcoming \emph{NASA Roman} telescope and the existing
    modeling methods are completely inadequate for dealing with the scale of data which will come
    from \emph{Roman}.
    I also propose a framework for dealing with issues which have plagued the field for
    decades -- various pathologies in microlensing models and questions about the interpretation
    of statistical results.
    Besides microlensing, I have also delved into the field of occultation mapping of Solar System
    objects and exoplanets.
    Together with collaborators, I have developed  a novel statistical method for reconstructing
    spatial maps of volcanic emission on Jupiter's moon Io using infra-red occultation
    light curves.
    I applied the same method to exoplanets to explore the exciting possibility of
    detecting weather changes on Hot Jupiters by reconstructing two dimensional maps of
    the emission surface from simulated \emph{JWST} secondary eclipse light curves.
    I found that planetary scale changes in the emission pattern should be detectable with
    \emph{JWST}.

\end{abstract}

\clearpage
\thispagestyle{empty}

% CHAPTER 1: Introduction
\chapter{Introduction}
\section{Context}
\section{Gravitational microlensing}
\section{Occultation and phase curve mapping}

% CHAPTER 2: The theoretical minimum
\chapter{The theoretical minimum}

\section{Gravitational microlensing}
\subsection{Deflection of light by gravity}
\subsection{The different scales of lensing}
%% Read the "scales of lensing" paper by that Italian guy
\subsection{Microlensing}
\subsection{The point mass lens and the lens equation}
\subsection{Parallax}
\subsection{Finite source effects}
\subsection{The binary lens}
\subsection{Multiple lenses}
\subsection{Solving the lens equation}

\section{Occultation and phase curve mapping}
\subsection{History}
\subsection{The Starry framework}
\subsection{Reflected vs. emitted light}

% In this and the following sections make sure to use microlensing and occultation
% mapping as examples
\section{Bayesian statistics}
\subsection{Probability theory}
\subsection{The meaning of probability}
\subsection{Probability as frequency of events in repeated trials}
\subsection{Probability as degrees of belief}
\subsection{The likelihood function}
% mention the likelihood principle
\subsection{Priors}
\subsection{On machine learning and the difference between explanation and prediction}
% mention that even if we have a model focused on inference, it might be best if we judge 
% its quality through its predictive performance

\section{Inference}
% check that paper on physics and ML by that MIT guy, there's a good stuff there. 
% check also Ian Murray's lecture notes  on MLPR
\subsection{The curse of dimensionality}
\subsection{There's no free lunch}
% exploration vs. explotation 
\subsection{Optimization vs. sampling}

\subsection{Sampling and optimization methods}
\subsubsection{Maximum likelihood}
\subsubsection{The Laplace approximation}
\subsubsection{Variational inference}
\subsubsection{Importance sampling}
\subsubsection{Rejection sampling}
\subsubsection{Markov Chain Monte Carlo}
\subsubsection{Hamiltonian Monte Carlo}
\subsubsection{Other notable methods}

\section{Model validation and comparison}
\subsection{Cross validation}
% discuss vanilla k-fold CV, stress out just how important and central CV is to science
\subsubsection{Pareto smoothed importance sampling LOO}
\subsection{Model comparison}
% don't do it!

\section{Putting it all together -- \emph{probabilistic programming}}
\subsection{The success of machine learning}
% that quote from Yann LeCun on differentiable code 
\subsection{Automatic differentiation}
\subsubsection{Forward mode autodiff}
\subsubsection{Reverse mode autodiff or backpropagation}
\subsubsection{Jacobian vector products}
\subsection{JAX}
\subsection{Numpyro}

% CHAPTER 3: Modeling microlensing events 
\chapter{Modeling microlensing events}
\section{Dealing with multimodal posteriors}
\section{A differentiable solver for the lens equation}
\section{Population level inference in microlensing}

% CHAPTER 4: Mapping the surface of Io 
\chapter{Mapping the surface of Io}
\section{The data}
\section{Information content}
\section{Pixel sampling}
\section{A static map model}
\section{A dynamic map model}

% CHAPTER 5: Mapping the surfaces of exoplanets
\chapter{Mapping the surfaces of exoplanets}
\section{Previous work}
\section{What can we learn using eclipse mapping?}
\section{Time dependent maps}

% CHAPTER 6: Conclusion
\chapter{Conclusion}
\section{Summary of the thesis}
\subsection{Microlensing}
\subsection{Occultation mapping}
\section{Future work}
\subsection{Microlensing in the era of the Roman telescope}
\subsection{Mapping volcanic activity on Io}
\subsection{Mapping exoplanets with JWST and beyond}

\appendix

\chapter{First appendix}
\chapter{Second appendix}

\end{document}