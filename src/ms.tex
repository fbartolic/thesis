\RequirePackage[l2tabu, orthodox]{nag} 
% Define document class
\documentclass[]{report}
\usepackage{style}

% Begin!
\begin{document}

% Title
\title{\vspace{-15pt}\begin{center}\includegraphics[width=0.4\linewidth]{static/misc/logo.jpg}\end{center}
\vspace{20pt}
\begin{center}
School of Physics \& Astronomy
\end{center}
\vspace{20pt}
\begin{center}
\LARGE\textbf{Probabilistic modeling of astrophysical time series:
gravitational microlensing and occultation mapping of planets and moons}
\end{center}
\vspace{20pt}
\begin{center}
Supervisor: Dr. Martin Dominik
\end{center}
}

\maketitle
\tableofcontents

% Abstract 
\begin{abstract}
    Here goes the abstract.
\end{abstract}

\clearpage
\thispagestyle{empty}

% CHAPTER 1: Introduction
\chapter{Introduction}
\section{Context}
\section{Gravitational microlensing}
\section{Occultation and phase curve mapping}

% CHAPTER 2: The theoretical minimum
\chapter{The theoretical minimum}

\section{Gravitational microlensing}
\subsection{Deflection of light by gravity}
\subsection{The different scales of lensing}
%% Read the "scales of lensing" paper by that Italian guy
\subsection{Microlensing}
\subsection{The point mass lens and the lens equation}
\subsection{Parallax}
\subsection{Finite source effects}
\subsection{The binary lens}
\subsection{Multiple lenses}
\subsection{Solving the lens equation}

\section{Occultation and phase curve mapping}
\subsection{History}
\subsection{The Starry framework}
\subsection{Reflected vs. emitted light}

% In this and the following sections make sure to use microlensing and occultation
% mapping as examples
\section{Bayesian statistics}
\subsection{Probability theory}
\subsection{The meaning of probability}
\subsection{Probability as frequency of events in repeated trials}
\subsection{Probability as degrees of belief}
\subsection{The likelihood function}
% mention the likelihood principle
\subsection{Priors}
\subsection{On machine learning and the difference between explanation and prediction}
% mention that even if we have a model focused on inference, it might be best if we judge 
% its quality through its predictive performance

\section{Inference}
% check that paper on physics and ML by that MIT guy, there's a good stuff there. 
% check also Ian Murray's lecture notes  on MLPR
\subsection{The curse of dimensionality}
\subsection{There's no free lunch}
% exploration vs. explotation 
\subsection{Optimization vs. sampling}

\subsection{Sampling and optimization methods}
\subsubsection{Maximum likelihood}
\subsubsection{The Laplace approximation}
\subsubsection{Variational inference}
\subsubsection{Importance sampling}
\subsubsection{Rejection sampling}
\subsubsection{Markov Chain Monte Carlo}
\subsubsection{Hamiltonian Monte Carlo}
\subsubsection{Other notable methods}

\section{Model validation and comparison}
\subsection{Cross validation}
% discuss vanilla k-fold CV, stress out just how important and central CV is to science
\subsubsection{Pareto smoothed importance sampling LOO}
\subsection{Model comparison}
% don't do it!

\section{Putting it all together -- \emph{probabilistic programming}}
\subsection{The success of machine learning}
% that quote from Yann LeCun on differentiable code 
\subsection{Automatic differentiation}
\subsubsection{Forward mode autodiff}
\subsubsection{Reverse mode autodiff or backpropagation}
\subsubsection{Jacobian vector products}
\subsection{JAX}
\subsection{Numpyro}

% CHAPTER 3: Modeling microlensing events 
\chapter{Modeling microlensing events}
\section{Dealing with multimodal posteriors}
\section{A differentiable solver for the lens equation}
\section{Population level inference in microlensing}

% CHAPTER 4: Mapping the surface of Io 
\chapter{Mapping the surface of Io}
\section{The data}
\section{Information content}
\section{Pixel sampling}
\section{A static map model}
\section{A dynamic map model}

% CHAPTER 5: Mapping the surfaces of exoplanets
\chapter{Mapping the surfaces of exoplanets}
\section{Previous work}
\section{What can we learn using eclipse mapping?}
\section{Time dependent maps}

% CHAPTER 6: Conclusion
\chapter{Conclusion}
\section{Summary of the thesis}
\subsection{Microlensing}
\subsection{Occultation mapping}
\section{Future work}
\subsection{Microlensing in the era of the Roman telescope}
\subsection{Mapping volcanic activity on Io}
\subsection{Mapping exoplanets with JWST and beyond}

\appendix

\chapter{First appendix}
\chapter{Second appendix}

\end{document}